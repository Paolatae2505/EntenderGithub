\documentclass[12pt]{report}
\usepackage{graphicx}
\usepackage{hyperref}
\usepackage[utf8]{inputenc}
\usepackage[spanish]{babel} 
\graphicspath{ {./PracticaNN/} }
  

\begin{document}
\title{Práctica 1: Opinión libro ``Una corte de espinas y rosas''}
\author{Vargas Bravo Paola}
\date{07 de Octubre 2020} 

\maketitle

\section*{¿Por que elegí este libro?}
Este es uno de mis libros favoritos es de la categoría de fantasía con Romance, es el inicio de una trilogía, me encanta hablar
de libros y me encanta hablar de esta trilogía de libros, es  una de mis espececialidades, y como ví que podíamos dar una opinión
de un libro me encantó esa idea y por eso decí hablar de ello.

\subsection*{Mi saga de libros favorita...}
Al ser mi saga de libros favorita antes que todo, ocuparé este espacio para de antemano recomendar este libro y a lo largo de este
documento daré mis puntos del libro y lo que más me gusto en general, es mi especialidad de hablar de libros que he leído y más especifico de este tipo de
libros que me encantaron.

\section*{¿De que trata el libro?}
La historia nos relata desde el punto de vista de Feyre una joven de 19 años que vive en un mundo que  esta divido
entre hadas y humanos, en verdad lo humanos tienen muy pequeño espacio del continente llamado Pyrithian, Feyre es la hermana mayor
de las tres hermanas,tiene que cazar para poder vender las pieles de los animales y obtener dinero de ello, ella es el único sustento de su familia, pues
su padre está inválido y su madre murió, la historia comienza cuando Feyre matá a un lobo gigante, que resulta ser un hada, eso hace que quiebre
los tratados entre humanos y hadas, así que se la llevan a el lado de las hadas para quedarse ahí para siempre, pero tal vez todo lo que
creía Feyre de las hadas no es verdad, decubre algo que no solo pone en peligro las cortes de las hadas, sino también el lado humano donde
están su padre y sus hermanas, que hará una humana en un mundo de hadas, las cuales ella odia.
 
\subsection*{Opinión}
Este es un libro genial, aunque debo decir que de los tres libros el segundo es mi favorito, aún así todos son geniales, el libro es un poco largo
tiene unas cuatrocientas páginas, es un libro que no es pesado, además te introduce a todo esto de los mitos de las hadas, en sí el primer libro
se centra en la maldición de la corte de la primavera, puesta por un hada llamada Amaranta, otro personaje principal es Tamlín el Alto lord
corte de la Primavera pues quien se la lleva a el lado de las hadas es el, y más especifico a la Corte de la primavera
, la maldicioń esta drenando la magia esto hace que el muro que separa hadas y humanos se debilite,ademas de que afecta a todas las cortes,
eso a su vez hace que las hadas queno son tan buenas salgan y ataquen, en general todo el libro te matiene en la historia,
en mi caso de verdad viví la historia hace mucho una maestra de literatura me dijo:
``Estarás leyendo realmente cuando vivas la historia y te pongas en la piel del personaje''.\\[1mm]
Decir que  todo se resulve en este libro sería una mentira, esto de verdad es solo es el inicio de un hilo que se extiende
y lo que cres que podría estar bien, ta lvez no es de esa forma aveces creemos que hacemos algo bien, por alguien o algo y aunque
sea de esa manera, eso no significa que por quien hacemos las cosas sea realmente la persona adecuada en nuestras vidas,
asi que este libro te enseña muchas cosas, que en el segundo te das cuenta que no todo tiene sentido, lo que tú creías
que esta bien, realmente no es de esa manera y lo entiendes todo. Este libro tiene demasiada fantasía para poder decir un poco de ello tengo que decir que
el lado de las hadas tiene Cortes y cada una de ellas tiene un Alto lord, dependiendo el tipo de corte tendrán cierto tipo de poderes, los que más me gustan
son los de la corte de la noche,todo se maneja con obscuridad, y hacen algo que le llaman ``tamización'' no es más que ir de un lugar a otro, una tipo
teletransportación pero con obscuridad, son muy geniales lo poderes que les ponen, son muchos más poderes, pero se centran más en los de
la corte de la primavera que es la transformación, más en especifico en lobos, así que podrán deducir que aquel lobo era de esta corte, acontinuación
todas las cortes  de las hadas con sus Altos Lords.\\[2cm]

\begin{tabular}{| c | c |}
  \hline
  Corte & Alto Lord \\ \hline
  Corte invierno & Kallias\\ \hline
  Corte Otoño & Beron \\ \hline
  Corte Primavera & Tamlin \\ \hline
  Corte Verano & Tarquin \\ \hline
  Corte amanecer & Thesan \\ \hline
  Corte Día & Helion \\ \hline
  Corte Noche & Rhysand \\ \hline
\end{tabular}

\paragraph{Personajes}
En sí todos los personjes son muy buenos, la protagonista es demasiado fuerte, no tiene ese tipo de personaje frágil ni mucho menos,
pero independientemente de la protagonista, Tamlin que es otro principal, no me gusto mucho como personaje, cree que todo lo que
hace esta bien  y no es necesariamente así, para mí hay un personaje mejor \textbf{Rhysand, es mi personaje favorito
  de este libro y en general de todos lo libros que he leído},
como lo vuelvo a decir todos son personajes muy completos y tienen cierta historia trágica y
en cuanto más avanza la historia crecen lo personajes de toda la historia
,pero vamos a hablar un poco de Ryshand:\\[0.2cm]

\textit{Rhyshand es el Alto Lord de la corte de la Noche, es un personaje, que lo tienen todo, desde mi punto de vista, te deja ver
muy poco de él en el primer libro pero conforme va avanzando la histotia, te deja ver mucho del personaje y del por que de sus acciones,
es un personaje, muy estructurado, con personalidad y los más importante es mi favorito, además con el tiempo se vuelve en el apoyo de la protagonista.}\\[0.2cm]

Es un libro muy bueno y creo que podriamos enlistar también a los personajes más imporatantes en la historia, ya que en el primer libro no son tantos,
hasta que no se desarolle poco a poco esta guerra interna en las hadas y después contra los humanos , que es bastante larga.

\begin{itemize}
\item Feyre
\item Tamlin
\item Cassian
\item Amaranta
\item Rhysand <3
\end{itemize}

También me gustaría decir realmente que este libro cuenta con un mapa para poder ubicar las cortes de las hadas
,el pequeño continte donde pasa todo es Phyritian, la verdad para mi es muy bueno que tenga este tipo de datos
los libros que te ubican en donde sucede todo el resto de la historia ayuda bastante, y siento que se capta bien la idea de la autora.\\[0.2cm]

\begin{figure}[h]
   \centering
   \includegraphics[scale=0.8]{cortes}
   \caption{Cortes de hadas}
 \end{figure}

\paragraph{Conclusión}
\texttt{Si podría decir una conclusión sería que realmente es un libro que me gusta mucho, me encanta el como manejan este tema de fantasía de las hadas,
el como la autora se dedico a hacer un mapa, con las cortes las cortes de las hadas que casí no se ven mucho más que la corte de Primavera,
pero despues se habla más de ellas, y auque todos crían que Amaranta era la amenaza principal, tal vez no sea así, pues ella absorbio todos o la mayor parte
de los poderes de todos los Lords de las cortes, es por ello que nadie puede darle pelea, hasta que se descubre una forma de romper la maldición
y la clave es Feyre, solo puedo decir que las pruebas que ella tiene que pasar para liberar a las hadas son muy fuertes, y ella aún así las paso
en esta especie de torneo de tres pruebas, a pesar de todos los obstacúlos que tiene en contra, y de que nadie crea en ella por ser humana aún
así ella logrará pasarlas y así poder liberar a todos de esta maldición y lo mas importante proteger a su familia, ya que el muro
se hace más fuerte mientras más magía tenga y así nada malo se pasa a al mundo humano, yo creo que es un libro que deberían leer si les gusta
la fantasía y romance, es un libro perfecto, por último también dejo un pequeño vincúlo para que si gustan  puedan saber un poco más de la historia
por que si vale la pena desde mi punto de vista}.\\[1mm]

\begin{figure}[h]
   \centering
   \includegraphics[scale=0.2]{ACCOTAR}
   \caption{Portada del libro}
\end{figure}

\url{https://una-corte-de-rosas-y-espinas.fandom.com/es/wiki/Una\_Corte\_de\_Rosas\_y\_Espinas}
\end{document}
